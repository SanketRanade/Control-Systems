\documentclass{beamer}
\mode<presentation>
\usepackage{adjustbox}
\usepackage{subcaption}
\usepackage{enumitem}
\usepackage{multicol}
\usepackage{listings}
\usepackage{url}
\def\UrlBreaks{\do\/\do-}
\usepackage{xcolor}
\definecolor{codegreen}{rgb}{0,0.6,0}
\definecolor{codeblue}{rgb}{0.8,0.3,0.3}
\definecolor{codegray}{rgb}{0.5,0.5,0.5}
\definecolor{codepurple}{rgb}{0.58,0,0.82}
\definecolor{backcolour}{rgb}{0.99,0.99,0.99}
 
\lstdefinestyle{mystyle}{
    backgroundcolor=\color{backcolour},  
    commentstyle=\color{codegreen},
    keywordstyle=\color{codeblue},
    numberstyle=\tiny\color{codegray},
    stringstyle=\color{codepurple},
    basicstyle=\ttfamily\footnotesize,
    breakatwhitespace=false,        
    breaklines=true,                
    captionpos=b,                   
    keepspaces=true,                
    showspaces=false,               
    showstringspaces=false,
    showtabs=false,                 
    tabsize=2
}
\lstset{style=mystyle}

\usepackage{hyperref}
\hypersetup{
    colorlinks=true,
    linkcolor=blue,
    filecolor=magenta,      
    urlcolor=cyan,
}

\urlstyle{same}
\usetheme{Ilmenau}
\usecolortheme{orchid}
\setbeamertemplate{footline}
{
  \leavevmode%
  \hbox{%
  \begin{beamercolorbox}[wd=\paperwidth,ht=2.9ex,dp=1ex,right]{author in head/foot}%
    \insertframenumber{} / \inserttotalframenumber\hspace*{2ex} 
  \end{beamercolorbox}}%
  \vskip0pt%
}
\setbeamertemplate{navigation symbols}{}


\theoremstyle{remark}
\newtheorem{rem}{Remark}
\newcommand{\sgn}{\mathop{\mathrm{sgn}}}

\lstset{
%language=C,
frame=single, 
breaklines=true,
columns=fullflexible
}

\numberwithin{equation}{section}

\title{Control Systems \\ Assignment 1}
\author{Sanket Ranade\\ EE19BTECH11012}
\date{\today} 

\begin{document}

\begin{frame}
\titlepage
\end{frame}

\section{Problem}
\begin{frame}
\frametitle{Problem Statement}
Q14. Use Python to generate the partial-fraction expansion of the following function
\begin{equation} 
F(s) = \dfrac{10^4(s+5)(s+70)}{s(s+45)(s+55)(s^2+7s+110)(s^2+6s+95)}
\label{}
\end{equation}
\end{frame}

\section{Python Code}
\begin{frame}{Python Code}
\lstinputlisting[language=python, firstline =1,lastline=14,basicstyle=\small]{control_assgn1.py}
\end{frame}

\begin{frame}
The following output we get while we run the above code:
\begin{figure}
\centering
\includegraphics[width=0.98\columnwidth]{output.png}
% \caption{Output}
\label{fig:circle_diameter}
\end{figure}
\vspace{10mm}
The above code is given here \url{https://github.com/SanketRanade/Control-Systems/blob/master/Assgn-1/control_assgn1.py}
\end{frame}

\end{document}
